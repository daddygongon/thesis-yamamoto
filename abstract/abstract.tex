\begin{abstract}

LPSO ( Long Period Stacking Order ) 構造をもった Mg は比降伏強度でジュラルミンを上回る特性を持ち, かつ難燃性であるため次世代の航空機の構造材料として国内外から注目を集めている. LPSO 構造は, 母相 hcp 構造の [0001] 方向に対して周期的に積層欠陥が導入されることで長周期性を有する構造である.

西谷研究室では, この LPSO 構造の生成機構として「積層欠陥部に L1$_2$ クラスターが形成され, そこから排斥された Zn, Y が, 濃化して新たな L1$_2$ クラスターを形成する」というシナリオを立てた. このシナリオの実現性について, 第一原理計算を用いて評価してきた. 第一原理計算は, 量子力学を支配するシュレディンガー方程式を精確に解いて, 原子の種類だけから電子構造を求め, いろいろな物性を予測する計算である. 計算の結果, 系全体のエネルギーは溶質原子と L1$_2$ クラスターとの距離が離れるにつれ単調に減少し安定となった. しかしそれは中周期的に溶質原子が濃化するという LPSO の構造から予想される結果に反するものであった.

本研究では, 「 Small Cluster と L1$_2$ クラスターの相互作用」および「空孔を含んだクラスターの安定性」に関して第一原理計算をおこなった.「 Small Cluster と L1$_2$ クラスターの相互作用」に関して計算をおこなった結果, 4 層から 5 層離れた位置でエネルギーが最安定である結果が得られた. この結果は Small Cluster が積層欠陥部から中距離離れた位置で安定化するというシナリオを支持した.「空孔を含んだクラスターの安定性」に関しては Small Cluster から 3 層離した位置に空孔を挿入したモデルが最安定であった. これは, Small Cluster の周りに空孔が吸着し, クラスター拡散が誘発されるという仮説に反するものであった.
 \end{abstract}