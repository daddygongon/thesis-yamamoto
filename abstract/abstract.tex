\begin{abstract}
LPSO ( Long Period Stacking Order ) 構造をもった Mg は比降伏強度でジュラルミンを上回る特性を持ち, かつ難燃性であるため次世代の航空機の構造材料として国内外から注目を集めている. LPSO 構造は, 母相 hcp 構造の [0001] 方向に対して周期的に積層欠陥が導入されることで長周期性を有する構造である. 西谷研究室では, この LPSO 構造の生成機構として「積層欠陥部に L12 クラスターが形成され, そこから排斥された Zn, Y が, 濃化して新たな L12 クラスターを形成する」というシナリオを立て, 第一原理計算を用いて系のエネルギーからそのシナリオの実現性を評価してきた. 第一原理計算は, 量子力学を支配するシュレディンガー方程式を精確に解いて, 原子の種類だけから電子構造を求め, いろいろな物性を予測する計算である. 計算の結果, 系全体のエネルギーは溶質原子と L12 クラスターとの距離が離れるにつれ単調に減少し安定となったが, それは中周期的に溶質原子が濃化するという LPSO の構造から予想される結果に反するものであった. 本研究では, 「 Small Cluster の配置の安定性」,「より大きなスラブモデルでの計算」および「空孔を含んだクラスターの安定性」をおこなう. 
 \end{abstract}